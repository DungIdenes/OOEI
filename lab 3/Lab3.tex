\documentclass[11pt, a4paper]{article}
\renewcommand{\baselinestretch}{1.5}
\usepackage[top=1.5cm,bottom=2cm]{geometry}
\usepackage[utf8x]{inputenc}
\usepackage{ucs}
\usepackage{amsmath}
\usepackage{amsfonts}
\usepackage[T1,T2A]{fontenc}
\usepackage{amssymb}
\usepackage{makeidx}
\usepackage[russian]{babel}

\begin{document}

Найти неопределенный интеграл

  $\int\sqrt{(x^2+1})$

Под корнем находится квадратный двучлен, и при попытке проинтегрировать данный пример чайник может мучаться часами. Такой интеграл берётся по частям и сводится к самому себе. В принципе не сложно. Если знаешь как.

Обозначим рассматриваемый интеграл латинской буквой   и начнем решение:

$ I=\int\sqrt{(x^2+1dx)}=(*)$

Интегрируем по частям:

$u=\sqrt{x^2+1}->du=xdx/\sqrt{x^2+1}$

$dv=dx->v=x$

$\int udv=uv- \int vdu$

$(*)=x\sqrt{x^2+1}-\int(x^2dx)/\sqrt{x^2+1}=x\sqrt{x^2+1}-\int(x^2+1-1)/\sqrt{x^2+1}=x\sqrt{x^2+1}0\-\int(\sqrt{x^2+1}-1/\sqrt{x^2+1})dx=x\sqrt{x^2+1}-\int\sqrt{x^2+1dx}+\int dx/\sqrt{x^2+1}=x\sqrt{x^2+1}-\int\sqrt{x^2+1dx}+ln|x+\sqrt{x^2+1}|$


(1) Готовим подынтегральную функцию для почленного деления.

(2) Почленно делим подынтегральную функцию. Возможно, не всем понятно, распишу подробнее:

$(x^2+1-1)/\sqrt{x^2+1}=(\sqrt{x^2+1}*\sqrt{x^2+1-1}))/\sqrt{x^2+1}=\sqrt{x^2+1}-1/\sqrt{x^2+1}$

(3) Используем свойство линейности неопределенного интеграла.

(4) Берём последний интеграл («длинный» логарифм).
 
Что произошло? В результате наших манипуляций интеграл свёлся к самому себе!

Приравниваем начало и конец:

$I=x\sqrt{x^2+1}-I+ln|x+\sqrt{x^2+1}|$

Переносим   в левую часть со сменой знака:

$2I=x\sqrt{x^2+1}+ln|x+\sqrt{x^2+1}|$

А двойку сносим в правую часть. В результате:

$I=1/2 x\sqrt{x^2+1}+1/2ln|x+\sqrt{x^2+1}|+C$

Или: $\int\sqrt{x^2+1}dx=1/2 x\sqrt{x^2+1}+1/2  ln|x+\sqrt{x^2+1}|+C$, где С = const

Константу  , строго говоря, надо было добавить ранее, но приписал её в конце. Настоятельно рекомендую прочитать, в чём тут строгость:

Примечание: Более строго заключительный этап решения выглядит так:

$…=x\sqrt{x^2+1}-\int\sqrt{x^2+1}dx+ln|x+sqrt{x^2+1}|+C$, где C=const

Таким образом:

$I=xsqrt{x^2+1}-I+ln|x+sqrt{x^2+1}|+C $

$2I=xsqrt{x^2+1}+ln|x+sqrt{x^2+1}|+C $

$I=1/2 xsqrt{x^2+1}+1/2ln|x+sqrt{x^2+1}|+C /2$

Константу$ C/2 $ можно переобозначить через  . Почему можно переобозначить? Потому что   всё равно принимает любые значения, и в этом смысле между константами  $C/2$ и   нет никакой разницы.

В результате:

$\int \sqrt{x^2+1}dx=1/2 xsqrt{x^2+1}+1/2ln|x+sqrt{x^2+1}|+C$, где C=const

Подобный трюк с переобозначением константы широко используется в дифференциальных уравнениях. И там я буду строг. А здесь такая вольность допускается мной только для того, чтобы не путать вас лишними вещами и акцентировать внимание именно на самом методе интегрирования.

\end{document}